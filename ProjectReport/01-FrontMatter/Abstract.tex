\renewcommand{\abstractname}{Resumé}
\begin{abstract}
  \noindent I 1980 blev det text-based adventure game Zork, udgivet for første gang. Spillet tillod
  udviklere at fortælle en kompliceret historie, påtrods af manglende 2D og 3D grafiske værktøjer.
  Ved at benytte de samme koncepter, som Zork brugte i 1980. Med brug af moderne udviklings metoder
  og værktøjer som agile, scrum, continuous integration og moderne backend og databaser har projektet
  udviklet sin egen udgave af en gammel indie klassiker.

  Gennem projektet blev det klar fra starten, at tidlig integration mellem projektets fire grene, frontend
  game engine, backend og database, var kritisk for projektets succes. Dette skyldes at processen når mange
  små ændringer integreres i projektet er simplere end proceseen der skal til at integrere store ændringer.

  Projektet følger den originale arkitektur ganske tæt, med kun få ændringer. Disse ændringer er der taget
  højde for senere i designet af projektets forskellige komponenter.

  Slut produktet er således en gennemtestet prototype til et spil, der med tiden kunne være blevet et fuld udbygget
  text-based adventure game, og det endte med at blive et gennemført proof of concept.
\end{abstract}

\begin{abstract}
\noindent 1980 saw the premier release of the text-based adventure game, Zork, which allowed 
the developers to tell a complex story despite the lack of advanced 2D and 3D graphics.
Utilising the same concepts as Zork in the 1980 with the addition of modern development methods
and tools like agile, scrum, a subset of continuous integration and a backend 
database solution, the projekt has successfully created its own take on an otherwise old
indie game genre. 

Throughout the projekt it became clear from early iterations that early integration 
between the projects four main branches, frontend, game engine, backend and database would be
critical for success as it simplified the final integration, when doing many small interations
instead of a few large once.

The project mostly follows the original architecture with only few deviations, that are otherwise
accounted for in the design og the projects many components. 

The end product is a well tested, prototype of what could later become a full fledged text-based
adventure game, and it turned out to be an excellent proof of concept.\\

\end{abstract}



\newpage