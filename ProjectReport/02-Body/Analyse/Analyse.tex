\section{Analyse}
Der vil i følgende afsnit blive præsenteret et kort udsnit af nogle af de analyser, der er blevet foretaget i projektet. Analyserne har til formål at danne et fornuftigt grundlag for videre udvikling af
projektet med mindskede risici. Der er foretaget teknologiundersøgelser, for at der kan gives det bedste teknologiske og metodemæssige bud på de respektive problemer der skal løses.

\subsection{Teknologi Undersøgelser}
Der er i teknologiundersøgelserne blevet undersøgt forskellige løsninger til diverse valg der skulle træffes tidligt i udviklingsfasen.\\
For det første skulle der træffes et valg om hvilket framework Frontend'en skulle udvikles i. Valget blev hurtigt indsnævret til to kandidater, nemlig .Net og Unity. Det blev fundet frem til, at med den 
opdeling af arbejdet, som gruppen ønskede at bruge, gav det mest mening at bruge et framework med en skarp opdeling af Front-end og Back-end. 
Dette kunne .Net opfylde. Derudover blev der i undervisningen på 4.semester undervist i .NET frameworket, således at der kunne fokuseres på at lave projektet og ikke lære frameworket at kende. 


\subsection{Teknologi Undersøgelse Backend}
Back-end’ens ansvar er at flytte og validere data mellem spillets frontend og spillets tilhørende Database. Da frontend'en udvikles i WPF vil det give mening at Backend ligeledes udvikles i et .NET miljø.
Der undersøges følgende to muligheder.\\

\subsubsection{Mulighed 1: WPF -\g ASP.NET Core -\g Entity Framework Core -\g Database}
Denne mulighed indebærer .NET frameworket ASP.NET Core. Her vil dataflowet altså blive følgende. Data’en skal flyttes først fra WPF til ASP.NET og derfra med EF Core\cite{Entity-Framework-Core} til databasen, og så tilbage igen.

\subsubsection{Mulighed 2: WPF -\g Entity Framework Core -\g database}
Dette er en mere simpel løsning, da det ikke kræver frameworket ASP.NET Core. Dataflowet vil her være direkte fra WPF til databasen ved hjælp af EF Core.
Her fåes nemlig muligheden for at separere data resourcerne fra resten af applikationen samtidig med vi får en mere sikker og pålidelig håndtering af brugerne.

Det besluttes at gå med mulighed 1 med ASP.NET Core, her fåes nemlig muligheden for at separere data resourcerne fra resten af applikationen samtidig med vi for en mere sikker og pålidelig håndtering af brugere.

\subsection{Teknologiundersøgelse: Database SQL/NOSQL
}
I dette afsnit diskuteres mulighederne for oprettelsen af en database. Heri vil der diskuteres fordele og ulemper ved anvendelsen af henholdsvis SQL eller No-SQL. Selve valget for databasen vil blive taget i design fasen.

\subsubsection{SQL}
Den første mulighed for systemets database er at anvende en SQL baseret database. Denne form for database
er relational, da den har en samling af data med forudbestemte forhold derimellem. Genstande
organiseres I tabeller som indeholder information om objektet. Yderligere er SQL kendt og vel
dokumenteret, som resulterer i at denne type database har et stort fællesskab bag den, med mulighed for online støtte til arbejdet med den. Ydermere fungerer SQL også med ACID-principperne.

\subsubsection{No-SQL}
Et andet alternativ ville være en database baseret på No-SQL. Denne har, modsat SQL, fleksible
datamodeller, som gør det nemmere at lave ændringer i databasen efterhånden som kravene til
databasen ændres. Yderligere har No-SQL også horizontal scaling, som betyder at hvis der rammes
den nuværende servers kapacitet, så har man muligheden for at tilføje mindre ekstra servers, hvor at man ville skulle migrere til en større server i SQL. Hernæst er NoSQL data ofte opsat efter queries. Sådan at data der skal hentes sammen,
ofte er opbevaret sammen. Dette resulterer i hurtigere queries.

\noindent Yderligere forklaring vedr. de teknologiske undersøgelser for databasen kan findes i teknisk bilag Sektion 7.2



