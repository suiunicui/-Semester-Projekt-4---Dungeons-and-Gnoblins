\subsection{Game Engine}
Game Engine er implementeret i C\#, et objectorienteret programmeringsprog
med stærk library support, der tillader at udvikle af applikationer. 
\autoref{fig:CoreDiagram} giver en klar insigt i hvordan kerne delen af
game engine er designet til at virke. 

\subsubsection{Game Controller}
Game controlleren er det centrale komponent i game engine; ansvarlig for 
kommunikation med frontend modulet. Game controlleren har adgang til alle
aspekter af spillet, gennem sin association med Combat controllern og 
backend controlleren.

\noindent Det er en nem løsning, og den fungere godt givet det begrænsede scope som 
game enginen skal fungere under.
For en mere kompliceret applikation med forventninger om udvidelser er det
dog ikke en god implementering da game controlleren har for mange grunde
til at ændre sig. Den har altså et for bredt ansvarsområde og kan formentlig
splittes op i flere klasser.

\noindent Når game controlleren instantieres laves et map af Map og Map creator klassen
hvilket før til en længere process hvor map creator klassen laver nogle
layout filer som map klassen kan læse for at danne et map i spillet, som 
spilleren kan navigere \parencite[Section 11.3.2][]{TekniskBilag}.\\

\noindent Game controlleren er i stand til at save og load games gennem 
dens relation til backend controlleren, der kan lave fetch til spillets
database.

\noindent Selve Load process er kompliceret, da den involvere adskillige iterationer
igennem alle rooms, items og fjender i spillet for at bekræfte om disse er
blevet besøgt, indsamlet eller beseret tidligere \parencite[Section 11.3.1]
[Figure 55]{TekniskBilag}.
Save game er simplere da game controlleren selv har adgang til alt infomation
som er nødvendigt for et komplet save game.

\subsubsection{Combat Controller}
Skulle spilleren befinde sig i combat, er det combat controlleren, der styre programmets
``flow of execution''. Først angriber spilleren; fjenden hvorefter fjenden angriber spilleren
hvis og kun hvis fjenden stadig er i live. Combat controlleren benytter en diceRoller til
at simulere terningekast, der benyttes til at bedømme om spilleren og fjenden rammer med
deres repektive angreb og efterfølgende; hvor meget deres angreb skader. Spilleren kan opsamle
og benytte våben til at forbedre deres terningekast \parencite[Section 10.3.2][]{TekniskBilag}. 

\subsubsection{Log}
Game Controlleren kommunikere med frontend gennem en log. Loggen logger forskellige
events når game controlleren ændre game statet. Frontend kan derefter benytte loggen
til at læse og display hvilket ændringer, der er foretage og præsentere det til spilleren
på en brugbar måde.

\newpage
