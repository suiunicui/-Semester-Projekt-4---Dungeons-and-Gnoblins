\newpage
\subsection{Diskussion af testresultater}
Projektets implementering og kravspecifikationerne har produceret adskillige tests, som 
projektet i overvejende grad har bestået. De fleste features er blevet
testet i henhold til kravspecifikationerne se \autoref{sec:kravspec} og har produceret 
resultater, der indikerer at hver feature fungerer som ønsket. \\

\noindent Relevante funktionelle tests er beskrevet med user-stories og er evalueret med et ``Godkendt''
eller ``Fejl'', som kan ses i i det tekniske bilag sektion 14.1 og evt.\ en foklarende kommentar. 
Ikke funktionelle 
test har fået en evaluering ``OK'' eller ``FEJL'' og kan findes i de tekniske bilag sektion 14.2.
Langt størstedelen af alle tests er bestået med få tests som fejler. 
Nogle tests fejler, da implementeringen ikke blev som forventet, mens andre ikke er blevet 
implementeret på grund af tidspres. \\

\noindent Et eksempel på dette er ``puzzles'' som skulle have været implementeret, som en del af det færdige spil.
Grundet tidspres er denne feature ikke blevet implementeret og
nedprioteret til fordel for andre features, såsom ``Combat'' der er blevet vurderet mere essentiel.\\

\noindent Et andet eksempel er``Delete Save Game'' som ikke blevet implementeret som specificeret i kravspecifikationerne, men
eksisterer som evnen til at overskrive allerede eksisterende save games.\\
Success raten i test skyldes en insistens på tidlige integration mellem alle store komponenter for at sikre, at 
alt kommunikationn har fungeret fra et tidligt 
tidspunkt i projektet. Det har vist sig nemmere at integrerer mange små ændringer i de individuelle moduler, end at lave en stor integration til sidst i udviklingsfasen.\\

\noindent Lad der dog ikke herske tvivl om, at der er huller i projektets test suit, det
betyder at store features ikke er testet i tilstrækkeligt omfang. Features som load- og save game er implementeret og testet ved visuel bekræftelse, men der er en betydelig mangel på modultest til yderligere at bekræfte, at disse feature fungerer som ønsket se \autoref{sec:kravspec}. \\

\newpage