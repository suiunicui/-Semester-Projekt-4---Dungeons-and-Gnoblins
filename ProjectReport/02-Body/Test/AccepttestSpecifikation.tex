\section{Accepttestspecifikation}
For accepttestspecifikationen er der for dette projekt, opstillet tests, og på baggrund af disse kan det betragtes som et funktionelt og acceptabelt produkt. 
Denne accepttestspecifikation er delt op i Funktionelle krav og ikke funktionelle krav.

\subsection{Funktionelle}
Ved de funktionelle krave er der her taget udgangspunkt i User Story 1,2,7 og 17, da disse indebærer alle systemets dele samt den mest centrale funktionalitet.\\
De resterende accepttest kan findes \textbf{REF HER}
\paragraph{Test af User Story 1 - Login}

\textbf{1. Scenarie - Succesfuld login}

\textit{Givet at brugerens profil er oprettet på databasen og at serveren er oppe.}

\begin{itemize}
  \item Når bruger indtaster sit login(Brugernavn og password)
  \item og trykker "Log in" på UI
  \item Så skifter skærmen til hovedmenu
\end{itemize}

Resultat: Godkendt\\
Kommentar:\\

\textbf{1. Scenarie - Fejlet login}

\textit{Givet at brugerens profil er oprettet på databasen og at serveren er oppe.}

\begin{itemize}
  \item Når bruger indtaster et forkert login(Brugernavn og password)
  \item og trykker "Log in" på UI
  \item Så signaleres der om forkert login-oplysninger til bruger
\end{itemize}

Resultat: Godkendt\\
Kommentar: Viser korrekt error besked\\

\paragraph{Test af User Story 2 - Opret profil}

\textbf{2. Scenarie - Bruger opretter profil}

\textit{Givet at brugerens profil er oprettet på databasen og at serveren er oppe.}

\begin{itemize}
  \item Når bruger vælger at oprette profil
  \item Så gemmes data for brugerens profil på databasen
  \item Og skærmen skiftes til log ind skærmen
\end{itemize}

Resultat: Godkendt\\
Kommentar:\\

\textbf{2. Scenarie - Bruger opretter samme profil}

\textit{Givet at brugerens profil er oprettet på databasen og at serveren er oppe.}

\begin{itemize}
  \item Når bruger vælger at oprette en allerede-eksisterende profil
  \item Så signaleres der om at profil allerede eksisterer
\end{itemize}

Resultat: Godkendt\\
Kommentar: Viser korrekt error besked\\

\paragraph{Test af User Story 15 og 18}

\textbf{15. Scenarie - Save Menu - Save Game}

\textit{Givet at brugeren har trykket "Save game" fra Settings menu og ikke er i combat.}

\begin{itemize}
  \item Når bruger vælger et gemt spil
  \item Og sætter navnet til det ønskede
  \item Og trykker på "Save Game" knappen
  \item Så genoptages spillet
\end{itemize}

Resultat: Godkendt\\
Kommentar:\\

\textbf{18. Scenarie - Main menu - Load Game - Load}

\textit{Givet at brugeren er logget ind og har adgang til spillet og trykket "Load Game"}

\begin{itemize}
  \item Når bruger vælger et gemt spil
  \item Og trykker "Load Game"
  \item Så loader spillet det gemte spil med det valgte game state
\end{itemize}

Resultat: Godkendt\\
Kommentar: Bruger starter i korrekt rum med korrekt inventory\\

\subsection{Ikke-funktionelle}
Herunder er udfyldte accepttest for systemets ikke-funktionelle krav. Her tages der igen et udpluk, som indeholder de mest kritiske krav. De resterende kan findes her\textbf{REF HERE}\\
\textbf{DATABASE}
\begin{table}[H]
\caption{ Fuldført ikke funktionelle tests for DATABASE}
\label{tab:}
\begin{tabular}{|p{3cm}|p{3cm}|p{3cm}|p{3cm}|}
\hline
Beskrivelse & Verificering & Resultat & Kommentar \\
\hline
Skal kunne gemme maksimalt 5 save games & Visuel & Fejl & Front-end begrænser antal gemte spil, databasen kan indeholde flere gemte spil end fem.  \\
\hline
Skal kunne loade et spil indenfor maksimalt 5s & Visuel & Godkendt &\\
\hline
Skal gemme hvilke genstande man bruger lige nu & Visuel & Godkendt & \\
\hline
Skal gemme hvor meget liv man har tilbage. & Visuel & Godkendt & \\
\hline
Skal gemme hvilke fjender man har slået ihjel. & Visuel & Godkendt & \\
\hline
Skal gemme hvilke puzzles man har løst & Visuel & Fejl & Puzzles er ikke implementeret \\
\hline
Skal gemme hvilke rum man har været i. & Visuel & Godkendt & \\
\hline
\end{tabular}
\end{table}

\textbf{GAMEPLAY}
\begin{table}[H]
\caption{ Fuldført ikke funktionelle tests for GAMEPLAY}
\label{tab:}
\begin{tabular}{|p{3cm}|p{3cm}|p{3cm}|p{3cm}|}
\hline
Beskrivelse & Verificering & Resultat & Kommentar \\
\hline
Spillets kort skal holde styr på hvilke rum man kan komme til for et givet rum. & Visuel & Godkendt & \\
\hline
Spillets kort skal kun vise de rum som spilleren har været i. & Visuel & Godkendt &\\
\hline
Spillets kort skal, hvis spilleren har været i alle rum vise alle rum. & Visuel & Godkendt & \\
\hline
Et rum kan have maksimalt 4 forbindelser til andre rum. & Visuel & Godkendt & \\
\hline
Et rum skal have mindst 1 forbindelse til andre rum. & Visuel & Godkendt & \\
\hline
Alle Rum skal kunne nås fra ethvert andet rum, måske ikke direkte, men man skal kunne komme dertil. & Visuel & Fejl & Tutorial rum er ikke tilgængeligt, når bruger har forladt det. \\
\hline
Spillerens rygsæk skal kunne indeholde alle spillets genstande. & Visuel & Godkendt & \\
\hline
Spilleren skal have mulighed for at bruge ét våben og ét skjold af gangen. & Visuel & Godkendt & \\
\hline
Spilleren skal have mulighed for at skifte hvilket våben og hvilket skjold der bruges. & Visuel & Godkendt & \\
\hline
\end{tabular}
\end{table}

\textbf{COMBAT}\\
\begin{table}[H]
\caption{ Fuldført ikke funktionelle tests for COMBAT}
\label{tab:}
\begin{tabular}{|p{3cm}|p{3cm}|p{3cm}|p{3cm}|}
\hline
Beskrivelse & Verificering & Resultat & Kommentar \\
\hline
Når spilleren/fjenden prøver at slå, rammer man kun hvis man på sit angreb slår højere end modstanderens rustningsværdi. Dette afgøres af et simuleret 20 sidet terninge kast, hvortil der lægges en værdi til, korresponderende til spilleren/fjendens våben bonusser. & Visuel & Godkendt & Et angreb går også igennem hvis slaget er lige på PC's rustningsværdi \\
\hline
Hvis spilleren/fjenden rammer, bliver skaden bestemt af et/flere simulerede terningekast, afhængigt af hvilket våben der bruges. & Visuel & Godkendt &\\
\hline
Hvis spilleren/fjenden rammer, bliver skaden bestemt af et eller flere simulerede terningekast, afhængigt af hvilket våben der bruges.  & Visuel & Godkendt & \\
\hline
Hvis spilleren, når nul liv inden fjenden, så dør spilleren og spillet er tabt. & Visuel & Godkendt & \\
\hline
Hvis fjenden, når nul liv inden spilleren, så dør fjenden og spilleren kan nu frit udforske rummet, som fjenden var i. & Visuel & Godkendt & \\
\hline
Hvis spilleren drikker en livseleksir bliver spillerens nuværende liv sat til fuldt & Visuel & Fejl & Livseliksir er ikke implementeret \\
\hline
\end{tabular}
\end{table}

\textbf{PERFORMANCE}
\begin{table}[H]
\caption{ Fuldført ikke funktionelle tests for PERFORMANCE}
\label{tab:}
\begin{tabular}{|p{3cm}|p{3cm}|p{3cm}|p{3cm}|}
\hline
Beskrivelse & Verificering & Resultat & Kommentar \\
\hline
 Spillet skal respondere indenfor maksimalt 5s & Visuel & Godkendt & \\
\hline
 Spillet må ikke have mere end én kommando i aktionskøen af gangen & Visuel & Fejl & Aktions kø er ikke blevet implementeret\\
\hline
\end{tabular}
\end{table}
