\subsection{Tidlig Integrationstest}
For gruppen var det et mål at komme igang med at lave integrationstest så hurtigt som muligt, da vi ønskede at der hurtigt kom styr på kommunikationen mellem de forskellige moduler i systemet. Her integreres altså blot vores \textbf{MVP REFERENCE HER}og det ville derefter være nemmere at implementerer nye features.\\

\noindent For at køre systemet har vi først startet den oprettede docker container til spillets database. Derefter startede vi spillets server, i form af backend api. Til slut kunne spillets klient åbnes og systemet testes.\\

\noindent I den tidlige integrationstest løb vi som gruppe ind nogle forskellige problemer.\\
Vi havde som gruppe arbejdet for opdelt i forskellige dokumenter. 
Derudover opdagede vi at de forskellige projekter benyttede forskellige versioner af .net, dette var dog hurtigt løst.\\
Til slut opdagede vi at der, når man loader et spil, ikke blev vist for brugeren hvilken rum man havde besøgt. Dette blev diskuteret og gruppen besluttede at tilføje dette til næste iterationer.\\

\noindent I \textbf{INDSÆT REF TIL VIDEO HER} ses en demovideo af integrationstesten af MVP. Her det ses at man kan spille spillet, gemme det og derefter loade det korrekt ind igen, dog igen uden at man kan se hvilke rum man har besøgt.
