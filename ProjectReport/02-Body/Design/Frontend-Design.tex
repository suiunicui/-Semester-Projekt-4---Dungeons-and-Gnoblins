\subsection{Frontend Design}
\label{ssec:FE Design}

Spillets front-end består af en række menuer, same selve spil viewet, som spillet skifter mellem i samme vindue.
Det primære spil-vindue er room view. Her præsenteres spilleren for en beskrivelse af det rum de er i, samt hvilke elementer i rummet de kan interagere med. Der vises også et kort over banen. Kortet Viser kun de rum spilleren allerede har været i, mens resten holdes skjult. Når brugeren så besøger et nyt rum, kan dette ses selvom spilleren forlader rummet. Dette lader spilleren udforske og oplåse hele kortet.\\
En række knapper nederst i højre hjørne på skærmen giver spilleren mulighed for at interagere med spillet. Fire knapper ("Go {North/West/South/East}") lader spilleren gå fra et rum til et andet. Ikke alle rum har forbindelse til alle sider, så det er f.eks. ikke altid muligt at trykke på "Go North". Kortet og rum beskrivelsen fortæller hvilken vej det er muligt at bevæge sig i.\\
Udover de fire retningsknapper er der et antal andre knapper. Disse bruges til at gemme spillet, gå til menuer, samt interagere med elementerne i rummet. Det specifikke antal og deres funktion er afhængig af den præcise implementering.

\begin{figure}[H]
\centering
\includegraphics[width = \textwidth]{02-Body/Images/RoomMockup.PNG}
\caption{Et mockup af det primære spil vindue. Tekst øverst i venstre side af skærmen giver en beskrivelse af det rum spilleren er i, samt en liste af elementer i rummet som spilleren kan interagere med. Øverst til højre vises et billede af banen. Spilleren interagerer med spillet via knapper nederst i højre hjørne. Knapperne "Go {North/West/South/East}" fører spilleren ind i et andet rum, mens de resterende knapper (markeret "Button") bruges til andre funktionaliterer i spillet.}
\label{fig:Design-FE-mockup-room}
\end{figure}