\newpage

\section{Indledning}
I 1980 udkom det første Text-based adventure game ”Zork” fra virksomheden Infocom (reference). Spillet blev en kæmpe succes, der stod for 20\% af virksomhedens salg i 1984, og var det mest solgte spil i 1982 (reference). Dette gav anledning til at lave en efterfølger på det populære spil, som udkom i det sene 80’ere. Her var spillet blevet udvidet med flere features.
Fra moderene forbrugeres synspunkt har Zork spillene nogle problemer. Blandt andet den manglende evne til at lave ”save games” som kan gemmes i cloud. Dette ville i dag gøre det muligt at kunne tilgå et specifik spil fra en anden enhed, uden at man skal flytte filer. Der var desuden ingen authentication, hvilket betyder at enhver der havde adgang til den pågældende PC, ville være i stand til at fortsætte eller slette gemte spil. 
Dette projekt vil forsøge at genopfriske den text-based adventure genre, ved at lave et nyt spil. Der vil blive taget inspiration fra Zork serien, ved at grundbasen for spillet, vil forblive nogenlunde det samme. Dog har projektet til hensigt at forsøge at tilføje nogle moderne aspekter til spillet - Heriblandt en opdaterende GUI der viser et map, en backend server til at save games og for authentication.


\subsubsection{Problemformulering}
Text based RPG er som sagt en spilgenre som var yderst populær i 80’erne. Målet med dette projekt er at genoplive spil genren Text based RPG, ved at udvikle et spil som kombinerer den klassiske spilgenre med ny teknologi. Problemet med tidligere Text based RPG spil, er at der har været begrænset mobilitet i forhold til at gemme og hente spillet på forskellige enheder. I forbindelse med denne problemstilling kan det være svært for brugeren at skabe sig et overblik over sine fremskridt i spillet efter en potentiel pause herfra. Hvad angår sikkerhed er det heller ikke noget der traditionelt har været fokus på i forbindelse med genren. På baggrund af disse udfordringer ved Text based RPG spil, vil dette projekt undersøge følgende problemstilling:

Er det muligt at genoplive den klassiske spilgenre Text-based RPG, igennem tilføjelse af moderne teknologier såsom cloudsaving, således at genren bliver attraktiv for den nye generation af spillere samtidigt med at den klassiske spiloplevelse fastholdes?
\\



