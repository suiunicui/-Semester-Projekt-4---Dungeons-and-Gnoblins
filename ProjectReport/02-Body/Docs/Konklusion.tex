\section{Konklusion}
Gruppen er endt ud med et funktionelt projekt, hvor de vigtigste features, blev implementeret.
Disse inkludere succesfuld load og save games, rum spilleren kan bevæge sig i, kamp mellem spilleren
og fjender, og en Frontend der gør spilleren istand til at inteagere med spillet.

Nogle features er ikke blevet implementeret, og er derfor sat til fremtidigt arbejde.
Nogle af disse features inkludere gåder og puzzles samt en evne til at slette save games.
Disse features er ikke anset som værende nødvendige for at opnå et funktionelt spil, som kunne spilles
igennem og stadig fungere som et spil med fjender og genstande, som kunne samles op og anvendes.\\

\noindent Fokus på kommunikation mellem systemets segmenter og integrationen af disse segmenter har været
et kerne mål gennem hele projektet og har leveret et spil hvor alle modulerne på succesfuld vis kan 
kommunikere med hinanden. 

Modulerne er lavet med fremtidigt arbejde i tankerne, så f.eks.\ i Game Module er det lavet relativt nemt 
at implementere yderligere features, så fremtidige features kunne implementeres uden at komprimere 
kommunikation mellem systemets segmenter eller med svage modifikationer dertil.

Modulerne er gennem testet og fungere som forventet, dog ikke til den fulde standard specificeret i 
continuous integration, men således at der god vished om at de fungere som forventet. Skulle projektet
udvides burde disse test omdannes til fuldt continuous integration for at sikre at nye features
inteageres på bedst mulig vis.\\

I sidste ende er slut produktet et glimerende eksempel på en tidlig prototype til et spil, der, givet mere
tid, kan blive et meget underholdene text-based adventure game.

\newpage
