\section{Fremtidigt Arbejde}
Sammenlignes det færdige produkt med det produkt der blev diskuteret under idefasen, er der nogle funktionaliteter som ikke er blevet færdiggjort, men kunne have været implementeret. Spillet kunne have haft flere udvidelser såsom Puzzles, flere niveauer, character udvikling og quest items som kunne have givet en bruger en mere underholdende oplevelse og mere avanceret gameplay. Dette hører ikke under et specifikt modul i systemet, men tilføjes over alle moduler, hvis nogen af de overstående features blev tilføjet. \\

\noindent F.eks. hvis der blev tilføjet flere niveauer, skulle der implementeres et nyt map i Game Controller og Front-end siden. Derudover skal der tilføjes en udvidelse af load og save game, så Back-end og database kunne gemme specifikt for det nye niveau, hvilket niveau er spilleren på, hvor langt brugeren er på det nuværende niveau og til sidst hvor meget spilleren har udforsket af tidligere niveauer.\\

\noindent Specifikt for Database og Back-End, kunne der tilføjes en lagring af data på en cloud-based storage fremfor lokal storage. Der blev valgt under forløbet at lagre dataen i en local storage, da der i midten af semestret var problemer med skolens licens af Microsoft. For ikke at komme ud for udfordringer senere i forløbet, blev der valgt at gå med den sikre løsning, at hoste databasen lokalt på enheden. Resultatet af et cloud-based lagring ville resultere i at spillet ville være mere tilgængelig for brugere på forskellige enheder eller platforme. \\

\noindent Derudover kunne spillet laves til at kunne køre på andre styresystemer som f.eks. IOS eller Unix. Dette ville lade en bredere målgruppe spille spillet og derved give en større kundegruppe. Denne ændring ville dog resulterer i at det valgte framework til Frontenden skulle skiftes til noget andet, såsom Unity. Dette ville betyde at frontenden og backenden skulle omskrives helt, men det kunne også give en generel bedre spil oplevelse.


