\subsection{Tidlig Integrationstest}
For gruppen var det et mål at komme igang med at lave integrationstest så hurtigt som muligt, da vi ønskede at der hurtigt kom styr på kommunikationen mellem de forskellige moduler i systemet. Det ville derefter være nemmere at implementerer nye features.\\

Før den samlede integrationstest, er der lavet mindre test af forskellige dele af systemet.
Frontend og Game engine er testet for sig og her kan man spille spillet, som i den tidlige integrationstest, blot er vores minimum viable product. \autoref{ssec:MVP}\\

Backend og database forbindelsen er også testet og der kan gennem swagger gemmes og hentes et save gennem api'ets get og post requests.\\

For at køre systemet har vi først startet den oprettede docker container til spillets database. Derefter startede vi spillets server, i form af backend api. Til slut kunne spillets klient åbnes og systemet testes.\\

I den tidlige integrationstest løb vi som gruppe ind i en række forskellige problemer.\\
Vi havde som gruppe arbejdet for opdelt i forskellige dokumenter. Blandt andet var DAL og backend api'et delt ud i forskellige projekter, som først måtte sammensættes og rettes før vi kunne gå videre.
Derudover opdagede vi at de forskellige projekter benyttede forskellige versioner af .net, dette var dog hurtigt løst.\\
Til slut opdagede vi at der, når man loader et spil, ikke blev vist for brugeren hvilken rum man havde besøgt. Dette blev diskuteret og tilføjet til de næste iterationer.\\

I \textbf{INDSÆT REF TIL VIDEO HER} ses en demovideo af integrationstesten af MVP. Her ses det at man kan spille spillet, gemme det og derefter loade det korrekt ind igen, dog igen uden at man kan se hvilke rum man har besøgt.
