\section{Konklusion}
Gruppen har endt ud med et funktionelt projekt, hvor de vigtigste features blev implementeret. Der er nogle features som ikke er blevet lavet og er derfor sat til fremtidigt arbejde. Ud fra projektet blev der valgt at disse features ikke var nødvendige for at opnå et funktionelt spil, som kunne spilles igennem og stadig fungere som et spil med fjender og genstande, som kunne samles op og anvendes.\\ Dette blev gjort da der var fokus på kommunikation mellem systemets segmenter og integrationen af disse segmenter i sidste ende. Modulerne er dog lavet med fremtidigt arbejde i tanken, så f.eks. i Game Module er det lavet relativt nemt at implementere yderligere features, så fremtidige features kunne implementeres uden at komprimere kommunikation mellem systemets segmenter.

\subsection{Personlige konklusioner}

\subsubsection{Luyen Vu}
Dette semesters projekt har været spændende og anderledes. Anderledes grundet at vi i dette semester ikke skulle arbejde med embedded software. Jeg vil argumentere for at vi i dette semester anvendte semestrets fag mere sammenlignet med de andre semestre, da vi som gruppe delte vores moduler op i de forskelige fags faggrupper. Derved har spørgsmål og troubleshooting været nemmere, idet at vi alle havde en ide om hvordan de forskellige moduler fungerede ift. en opdelling af Software- og elektronikstuderende. 
Vores produkt er endt fint ud i min optik og det har været et interessant projekt hele vejen igennem. Det faktum at det kun var software har også haft en indflydelse på mit syn af interessen af projektet.

\subsubsection{Oscar Dennis}
Projektet dette semester fungerede på en anden måde en projekter, som man har været vant til. Dette var på baggrund af at dette projekt ikke kravede samarbejde med elektronik ingeninørerne, og var derfor et rent software baseret projekt. Dette resulterede i at da vi alle var på samme studie, så var det nemmere at forstå og reviewe andres arbejde. Yderligere opstod der problemer med integration af de forskellige segmenter, da grupperne, såsom Front-end, Back-end og database, arbejde meget individuelt og opdelt. Derfor var der adskellige problemer som skulle løses før systemet kunne sættes sammen og fungere.\\
Dog har gruppen endt ud med et funktionelt projekt som jeg personligt er meget tilfreds med.



\subsubsection{Magnus Blaabjerg Møller}

\subsubsection{Morten Høgsberg}
Dette er det første projekt uden restriktioner på projektet emne. Projektet har således for første gang
være et rent softwareprojekt. Denne gang har ideèn været at implementere et spil, hvilket er gået over
alt forventning. Arbejdsmoralen har været høj igennem projektet, med input fra alle medlemmer, og en
villighed til at lytte og give konstruktivt feedback til hinanden.
Det har været nemmere at følge up på hinandens arbejde da vi alle har haft en basis forsåelse af hvordan 
de andre har udført deres opgaver. I sidste ende har vi leveret i projekt, som opnådede at implementere
de fleste features og som fungere bedre end forventet. 
