\subsection{Teknologiundersøgelse: Unity - .Net}

I projektgruppen har vi identificeret at der er to oplagte frameworks, til at lave projektet i.
Det første framework er Microsofts "WPF" til frontend og "ASP.NET" til systemets backend.
Det andet framework er Unity Technologies' framework "Unity" til kombineret frontend og backend.
Ens for de to framework er at de begge understøtter blandt andet:
- Understøtter C\# 
- Versionskontrol i form af Git
- Integrerede/let integrerbare Testframeworks

\subsubsection{Unity:}

Unity lader dig bygge fulde spil fra bunden, dette giver rig mulighed for udvidelse og ændringer af spil designet, dette kombineret med 
at Unity er bygget til cross-platform kompilering, hvilket gør at ens spil automatisk bliver portabelt og ens målgruppe bliver udvidet og derved
bliver markedsføring af produktet mere bred. Dette har naturligvis en pris, i form af at en Unity licens til firmaer med en årlig omsætning 
på over 100.000\$, koster minimum 1.800\$.
Derudover er Unity umiddelbart intuitivt og nemt at bruge, med det menes der at Unity's frontend er informativ, 
dog er hele frameworket nyt og skal derfor læres fra bunden. Dette resulterer i at der skal undersøges mange basale ting for at der kan laves et spil,
og tiden det tager at lave første iteration kan hurtigt eksplodere. Unity er også et program i aktiv udvikling og det betyder
for det første at features hele tiden bliver udviklet og udvidet. Dette resulterer også i at dokumentationen omhandlende Unity
hele tiden skifter og derfor kan det være svært at søge hjælp på nettet, når man sidder fast i udviklingen.
Unity har udover sit udviklingsværktøj, en Asset store, som giver mulighed for at købe allerede programmerede Assets og trække dem direkte ind
i ens program. Dette giver selvfølgelig mulighed for at spare en masse arbejde, men da projektarbejde netop handler om at lave tingene selv,
giver dette ikke meget mening at bruge i dette tilfælde.
Et negativ ved at bruge et så gennemført og feature rigt udviklingsværktøj, er at kompilerings, loading og test tider bliver markant højere end
ved brug af lettere udviklingsværktøjer.
Til sidst skal der nævnes at udvikling i Unity ikke bliver opdelt i frontend/backend, sådan som det bliver lært at udvikling skal opdeles på 4. semester,
Tværtimod sidder frontend/backend meget tæt sammen, når man udvikler med Unity.\cite{Unity.com} \cite{Choose-Unity} \cite{Unity-Pros-Cons}



\subsubsection{.NET framework:}
Til forskel fra Unity, er .NET Frameworket noget mere letvægt. Spillet kan stadig laves fra bunden og stadig med rig mulighed for ændringer og udvidelse af spil designet. Hvad man ikke får, er Unity's indbyggede Cross-platform kompilering og derved kan denne del af ens udvidelses fase godt blive mere kompliceret, hvis man vil kompilere til flere styresystemer. At man misser den indbyggede Cross-platform kompilering, og Unity's Asset store er også med til, som sagt, at gøre .Net frameworket mere letvægt, det vil sige at man kan se frem til mindre overhead og derved potientielt hurtigere load og compile tider.
Kigger man på de mere økonomiske forskelle, ser man hurtigt at de to frameworks er som to modsætninger, nemlig fordi at .NET frameworket kan fås gratis som extension til Visual Studio, som der også findes gratis versioner af.  
Ser man på .NET frameworket gennem mere praktiske øjne for os som studerende, dukker der umiddelbart tre argumenter frem.
1. Vi ved fra undervisningen, hvor vi har arbejdet med netop .NET frameworks, at der er massere af kendt viden at hente, altså dokumentation at hente på nettet og fra vores undervisere, om hvad og hvordan vi bedst kan gribe lige netop dette framework an.
2. Frontend og Backend er i .NET framworket separeret, hvilket passer rigtig godt med vores projektarbejde i projektgruppen og med hvordan vi har lært at man skal udvikle software systemer i undervisningen, nemlig lige præcis med den separerede frontend og back end.
3. Vi har også tidligere arbejdet med hvordan CI (Continious Integration) kan sættes op, netop sammen med .NET frameworket.
En opsumering af de sidste 3 punkter giver altså at vi i .NET framework, på trods af at det er mere letvægt end unity, kan fokusere mere intenst på at lave noget godt projektarbejde og kode, og ikke fokusere så meget på de basale ting, som vi måske skulle hvis vi arbejdede med Unity.\cite{Microsoft.com}

