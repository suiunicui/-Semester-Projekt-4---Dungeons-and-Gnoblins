\subsection{Frontend Arkitektur}

Frontend applikationen vil have til formål, at håndtere brugerinput og output. Dvs. at der i Frontenden vises det data fra gamelogic, som brugeren skal have, og at det præsenteres på en overskuelig og brugervenlig måde. Dette resulterer i at brugeren kan forstå og kan finde ud af at bruge spillet på den tiltænkte måde.
Derudover skal Frontenden tage hånd om bruger input, og sørge for at brugeren giver korrekt input og at der tages hånd om eventuelt forkert input.

\subsubsection{Pseudo Frontend Arkitektur}
(Hovedrapport stuff)
For at give overblik over, hvordan kommunikationen mellem frontend, backend og gamecontroller kommer til at foregå, er der lavet et pseudo sekvensdiagram for UserStories:
- Login
- Register
- Save Game
- Load Game
(/Hovedrapport stuff)

Der er ikke lavet sekvensdiagrammer for alle af projektets userstories, da mange af disse fungerer på samme måde og derfor ikke bidrager med noget nyt ift. dokumentationen.
Nedenfor ses pseudo sekvensdiagrammer for de fire userstories:
- Login
- Register
- Save Game
- Load Game

\begin{figure}[h]
\centering
\includegraphics[width = \textwidth]{02-Body/Images/RoomMockup.PNG}
\caption{Et mockup af det primære spil vindue. Tekst i venstre side af skærmen giver en beskrivelse af det rum spilleren er i, samt en liste af elementer i rommet som spilleren kan interagere med. Øverst til højre vises et billede af banen. Spilleren interagerer med spillet via knapper i højre hjørne. Knapperne "Go {North/West/South/East}" fører spilleren ind i et andet rum mens de resterende knapper (markeret "Button") bruges til andre funktionaliterer i spillet.}
\label{fig:Design-FE-mockup-room}
\end{figure}


