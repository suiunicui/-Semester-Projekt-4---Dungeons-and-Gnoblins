\subsection{Diskussion af testresultater}
Projektets implementering og kravspecifikationerne har produceret adskillige tests, som 
projektet i overvejende grad har bestået. De fleste features er blevet
testet i henhold til kravspecifikationerne se \autoref{sec:kravspec} og har produceret 
resultater, der indikerer at hver feature fungerer som ønsket. \\

Alle funktionelle tests er beskrevet med user-stories og er evalueret med et ``Godkendt''
eller ``Fejl'' \autoref{tab:testEngine} og evt.\ en foklarende kommentar. Ikke funktionelle 
test har fået en evaluering ``OK'' eller ``FEJL''.
Langt størstedelen af alle tests, for produktet, er bestået med få tests som fejler. 
Nogle tests fejler, da implementeringen ikke blev som forventet, mens andre ikke er blevet 
implementeret på grund af tidspres. \\

To eksempler er ``puzzles'' og ``Delete Save Game'', der skulle have været implementeret, som en del af det færdige spil.
Grundet tidspres er disse features ikke blevet implementeret, men i stedet for, er de blevet
nedprioteret til fordel for andre features, såsom ``Combat'' der er blevet vurderet mere essentiel.
``Delete Save Game'' er ikke blevet implementeret som specificeret i kravspecifikationerne men
eksisterer i stedet for, som evnen til at overskrive allerede eksisterende save games. Her er 
kravende ikke blevet opdateret til at reflektere den anderledes implementering. Yderligere er MTBF heller ikke blevet testes da dette vil være besværligt at teste for denne type projekt.\\

Den stores success rate skyldes en insistens på tidlige integration mellem alle store komponenter for at sikre, at 
alt kommunikation mellem frontend, game engine, backend og databasen har fungeret fra et tidligt 
tidspunkt i projektet. I stedet for at integrere og teste alting samtidigt, er hvert komponent 
blevet gradvist integreret i projektet. Det har vist sig nemmere at integrerer mange små ændringer
end at lave en stor integration til sidst i udviklingsfasen.

Lad der dog ikke herske tvivl om, at der er store huller i projektets test suit, det
betyder at store features ikke er testet i tilstrækkeligt omfang. Features som load- og save game er implementeret og testet ved visuel bekræftelse, men der er en betydelig mangel på modultest til yderligere at bekræfte, at disse feature fungerer som ønsket (se \autoref{sec:kravspec}). \\
