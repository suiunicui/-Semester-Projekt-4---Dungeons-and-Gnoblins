\section{Projektforslag}

I 1980 udkom det først Text-based adventure game Zork I
fra virksomheden Infocom\cite{WikiZork}. Spillet var en
kæmpe succes, der stod for 20\% af virksomhedens salg i
1984 og var det mest solgte spil i 1982\cite{WikiZork}.

I 1987 bliver den sidste udgave i Zork serien, med titlen
Beyond Zork, udgivet\cite{WikiBeyondZork}.%
Spillet var blevet udvidet med  ``procedual world generation and character creation''\cite{WikiBeyondZork}.

\vspace{0.7em}

\noindent Et ledende problem med Zork spillene set fra en moderne 
forbrugers synspunkt var zorks manglende evne til at lave
``save games'' der er gemt spillet i cloud, hvilket idag tillader
at et spil kan genoptages på en anden enhed uden at skulle
flytte filer fra en PC til en anden gennem USB eller andet hardware. 

Zork spillene tillod ikke et map over ens position relative til
de opdagede områder i spillet. Spilleren var således selv ansvarlig
for at holde styr på deres placering i et spil kun baseret på 
tekst outputs.

Zork serien benyttede ingen authentication til at forhindre andre
bruger af en PC fra forsætte eller slette gemte spil.

\vspace{0.7 em}

\noindent I Moderne tid har de største spil producenter focus 
på multiplayer spil, League of Legends, CS:Go, Minecraft osv.\
spil uden et ``Story Element'' på trods af at største delen
af forbrugerne foretrækker singleplayer spil\cite{Statista}.

\vspace{0.7em}

\noindent Projektet søger at opfriske den text-based adventure genre
ved at lave et ny spil.
Spillet søger inspiration fra Zork serien, ved at kigge på
de gode aspekter af spillene, heriblandt dens revolutionerende
parser, der satte spilleren i stand til beskrive sine
aktioner med sætninger i steddet for to ords kommandoer f.eks.
``I place the lamp and sword in the bag'' istedet for ``take sword''\cite{WikiZork}.

Projektet kigger også på muligheder for at lave verdenen Procedually generated,
således at spillets verden er forskellig efter at brugeren starter
et nyt Spil. Målet er at øge re-playability for spillet,
noget først forsøgt i Beyond Zork\cite{WikiBeyondZork}.

Samtidig forsøger projektet af tilføje følgene aspekter til genren --%
en opdatere GUI, med et map over spilleren position, en backend server
til save games og accounts med authentication. 

\newpage
